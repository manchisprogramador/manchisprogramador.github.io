\documentclass{article}
\usepackage{amsmath,amsfonts}
\usepackage{algorithmic}
\usepackage{array}
\usepackage{nameref}
\usepackage[caption=false,font=normalsize,labelfont=sf,textfont=sf]{subfig}
\usepackage{circuitikz}
\usepackage{textcomp}
\usepackage{stfloats}
\usepackage{url}
\usepackage{verbatim}
\usepackage{graphicx}
\usepackage{balance}
\usepackage{setspace}
\usepackage{multirow}
\usepackage{float}
\def\tablename{TABLA}
\usepackage[breaklinks]{hyperref}
\hypersetup{hidelinks}
\usepackage[table]{xcolor}
\usepackage{cite}
\renewcommand\refname{Referencias}
\usepackage{amsmath}
\usepackage{tikz}
\usetikzlibrary{shapes, arrows.meta, positioning}
\usepackage[pages=all]{background}
\backgroundsetup{
    scale=1,color=black,opacity=0.6,angle=0,
    hshift=0mm,
    vshift=0mm,
    contents={\includegraphics[width=210mm, height=298mm]{plantilla_gl.pdf}} % Ajusta ancho y alto aquí
}
\usepackage{csquotes}%paquete para comilla y algo mas 
  % Para imágenes de fondo
\usepackage{tcolorbox}
\usepackage[absolute,overlay]{textpos}
\TPGrid{16}{24} % Define una cuadrícula de 16x24 (ancho x alto) en la página

%definiendo colores personalizados
\definecolor{lightgray}{gray}{0.9}  % Gris claro
\definecolor{lightblue}{rgb}{0.8, 0.85, 1.0}  % Azul claro
\definecolor{lightgreen}{rgb}{0.85, 1.0, 0.85}  % Verde claro
\definecolor{salmon}{rgb}{1.0, 0.6, 0.6}  % Rosa salmón
\definecolor{lightyellow}{rgb}{1.0, 1.0, 0.8}  % Amarillo claro
\definecolor{peach}{rgb}{1.0, 0.9, 0.7}  % Melocotón (peach)




\usepackage[a4paper,top=5.5cm,bottom=2cm,left=2.5cm,right=2.5cm]{geometry} % Ajusta los márgenes

\begin{document}

\begin{titlepage}
    \begin{center}
        
        
    \end{center}
    \vspace{15mm}
    
    \begin{center}
        {\LARGE{\bf Dudas sobre la oferta presupuestal}}\\
        
        \LARGE{\textbf{OP-240904-14616543-SR-3509172 Grupo lopera}}
    \end{center}
    
    \vspace{40mm}
    \par
    
    \noindent
    \begin{minipage}[]{1\textwidth}
    \centering
        \large{
            % \bf Relatore: \\ % IT
            \bf Encargado: \\ % EN
            Ing. German Mora Mora 
        }
    \end{minipage}
    \hfill
    
    
    \vspace{40mm}
    
    \begin{center}{
        \large{\bf\today}
        }
    \end{center}
\end{titlepage}
   
   
\section*{Consideraciones}


\section*{Alcance Técnico}

Entre los \textbf{POI} y \textbf{Número máximo de procedimientos} solicitamos amablemente un ejemplo práctico para la \textbf{Universidad tecnológica de Pereira} y para \textbf{GL ingenieros} donde se pueda identificar los \textbf{POI´s} y el \textbf{Número máximo de procedimientos} esto con el fin de poder dimensionar lo mejor posible el alcance de proyectos tanto educativos como industriales.

\subsection*{\textbf{Máximo de usuarios}}

Sobre este ítem surge la duda si cada uno de los usuarios puede ingresar desde cualquier dispositivo, y si pueden estar conectados \textbf{todos al mismo tiempo} tanto para la versión \textbf{educativa} como para la \textbf{industrial}

\subsection*{\textbf{Soporte remoto}}

La duración en horas de este soporte \textbf{(4 horas de soporte remoto por parte de un experto de XROA)} son durante todo el año que dura la licencia, es decir, en un año solo se cuenta con cuatro horas de soporte remoto por parte del experto en XROA?

\subsection*{\textbf{Capacitación de edición remota}}

De igual manera que el ítem anterior, solo se cuenta con 4 - 6 horas durante todo el año de licenciamiento en \textbf{capacitación de edición remota?}

\section*{Solicitudes}

De ser posible una reunión a través de la plataforma \textbf{Meet} o la que se considere pertinente, análoga o similar a la mencionada para el cierre de dudas e inquietudes mencionadas a lo largo de este documento. 

lo que sea necesario para aclarar las dudas mencionadas en este documento me alegro

  
    
\end{document}